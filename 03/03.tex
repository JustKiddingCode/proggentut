\documentclass[18pt]{beamer}
%% SLIDE FORMAT
\usepackage[utf8]{inputenc}
\usetheme{Rochester}

\title[Programmieren Tutorium]{3. Programmieren Tutorium:\\ Von Konstanten und Referenzen}
\subtitle{Datentypen / Referenzen / Datentypen und Schinken}
\author{Konstantin Zangerle \\ info@konstantinzangerle.de}

\usepackage{listings}
\usepackage{color}

\definecolor{mygreen}{rgb}{0,0.6,0}
\definecolor{mygray}{rgb}{0.5,0.5,0.5}
\definecolor{mymauve}{rgb}{0.58,0,0.82}

\lstset{ %
  backgroundcolor=\color{white},   % choose the background color
  basicstyle=\footnotesize,        % size of fonts used for the code
  breaklines=true,                 % automatic line breaking only at whitespace
  captionpos=b,                    % sets the caption-position to bottom
  commentstyle=\color{mygreen},    % comment style
  escapeinside={\%*}{*)},          % if you want to add LaTeX within your code
  keywordstyle=\color{blue},       % keyword style
  stringstyle=\color{mymauve},     % string literal style
}


\begin{document}

% change the following line to "ngerman" for German style date and logos
\selectlanguage{ngerman}

%title page
\begin{frame}
\titlepage
\end{frame}

%table of contents
\begin{frame}{Gliederung}
\tableofcontents
\end{frame}




%% Konstantendeklaration und null
\section{Konstantendeklaration und null}
\begin{frame}{Konstantendeklaration und null}
\begin{itemize}
 \item final: Der Wert dieser Variable darf sich nicht ändern.
 \item null: Kein Objekt!
\end{itemize}
\end{frame}



\section{Parameter}
\begin{frame}{Parameter}
Ein Parameter ist eine besondere Form von Variablen
Wir unterscheiden:
\begin{itemize}
 \item \textbf{Formaler Parameter}: \pause \\
  Der Name, der bei der Deklaration der Konstruktor Methode gewählt wurde.
  
 \item \textbf{Aktueller Parameter}:\pause \\
 Der Wert, der beim Aufruf für den formalen Parameter eingesetzt wird.
\end{itemize}
\end{frame}


\section{Konstruktoren}
\begin{frame}{Konstruktoren}
 Haben wir letzte Woche nocheinmal behandelt.
 Deswegen heute nur eine Aufgabe dazu!
 
 \textbf{Aufgabe:} Modelliere eine Klasse Quadrat, die als Konstruktoren die obere linke Ecke und die untere rechte Ecke annimmt.
 Ebenso soll es auch einen Konstruktor geben, der alle 4 Eckpunkte annimmt und auf Korrektheit überprüft, sowie einen Konstruktor
 der die obere linke Ecke sowie zusätzlich eine Länge annimmt.
\end{frame}

\section{enum}
\begin{frame}{enum}
 Oft will man in einem Programm eine Auswahl bereitstellen. Es soll also eine Klasse geben,
 die es erlaubt einen begrenzten Wertebereich anzugeben.
 Bsp.: Für ein Programm, dass mit Monaten arbeitet ist es blöd, integer als Parameter zu verwenden. Das erfordert immer eine
 zusätzliche Überprüfung! \\
 \pause
 \textbf{Lösung: } ENUM!
\end{frame}
\begin{frame}[fragile]{enum - Beispiel}
 \begin{lstlisting}[language=java]
  public enum Months {
    JAN, FEB, MAR, APR, MAI, JUN, JUL, SEP, OCT, NOV, DEZ
  }
 \end{lstlisting}
\end{frame}

\begin{frame}{enum - Aufgabe}
 Sicherlich bekannt sind die Lieder: 
 \begin{itemize}
  \item Manic Monday
  \item Ruby Tuesday
  \item Wednesday's Song
  \item Thursday
  \item Friday I'm in love
  \item Saturday night fever
  \item Sunday Morning
 \end{itemize}
 Schreibt ein Programm, dass per Eingabe eines enum Typs mir den richtigen Song ausgiebt. 
 
 (Bonuspunkte für den richtigen Interpreten)
\end{frame}

\begin{frame}{enum - Aufgabe}
 Sicherlich bekannt sind die Lieder: 
 \begin{itemize}
  \item Manic Monday – Bangles
  \item Ruby Tuesday – Rolling Stones
  \item Wednesday's Song – John Frusciante
  \item Thursday – Pet Shop Boys
  \item Friday I'm in love – The Cure
  \item Saturday night fever – Bee Gees
  \item Sunday Morning – No Doubt
 \end{itemize}
 Schreibt ein Programm, dass per Eingabe eines enum Typs mir den richtigen Song ausgiebt. 
 
 (Bonuspunkte für den richtigen Interpreten)
\end{frame}

\section{Methoden}
\subsection{Methodensignatur}
\begin{frame}[fragile]{Methoden}
 \begin{lstlisting}[language=java]
  public int convertToFahrenheit(int celsius) {
    return (celsius * 9) / 5 + 32;
  }
 \end{lstlisting}
 ist Methode der Klasse Temperatur. Die Methodensignatur ist der Fingerabdruck der Methode für den Compiler.
 
 Frage: Aus was besteht dieser Fingerabdruck?
\end{frame}

\begin{frame}{Methodensignatur}
Antwort:
 \begin{itemize}
  \item Namen der Methode
  \item Anzahl der Parameter
  \item Reihenfolge der Parameter
  \item Typen der Parameter
  \item Rückgabetyp der Methode
 \end{itemize}

\end{frame}

\subsection{Überladen von Methoden}
\begin{frame}{Überladen von Methoden}
 Methoden können den gleichen Namen haben – solange die Signatur unterschiedlich ist, weiß der Compiler/Interpreter was er nehmen soll.
 Ihr habt heute schon Methoden überladen!? \\ \pause
 \textbf{Genau! Den Konstruktor}
\end{frame}

\begin{frame}[fragile]{Überladen – noch ein Beispiel}
 \begin{lstlisting}[language=java]
public class Ueberladen {
    public int max(int i1, int i2) { 
        return (i1 > i2) ? i1 : i2; 
    }
   
    public float max(float f1, float f2) { 
        return (f1 > f2) ? f1 : f2; 
    }
    
    public double max(double d1, double d2) { 
        return (d1 > d2) ? d1 : d2; 
    }
} 
 \end{lstlisting}
\end{frame}

\section{static – Was ist das?}
\begin{frame}{static was ist das?}
 static zeigt an dass etwas – Methode oder Attribut – nicht zum Objekt sondern zur Klasse, d.h. für alle Instanzen der Klasse genau einmal existiert.
 Was bedeutet das für uns?
\end{frame}

\begin{frame}{static – die Auswirkungen}
\begin{itemize}
 \item eine static Funktion kann keine nicht-static Attribute benutzen
 \item eine static Funktion kann keine nicht-static Funktion benutzen
 \item Umgekehrt funktioniert aber beides!
 \item Kann für Konstanten verwendet werden
\end{itemize}
 

\end{frame}


\begin{frame}[fragile]{Konkrete for – macht ihr Selbst!}
 \begin{lstlisting}[language=java]
 System.out.println("Einmal");
 int i = 10;
 while(i > 0) {
  System.out.println("Und nocheinmal...");
  i--;
 }
 \end{lstlisting}
 Aufgabe: Wandle obige WHILE-Schleife in eine FOR-Schleife um!
\end{frame}

\begin{frame}[fragile]{Sollte aber ungefähr so aussehen!}
 \begin{lstlisting}[language=java]
 System.out.println("Einmal");
 for (int i = 10; i > 0; i--) {
   System.out.println("Und nocheinmal...");
 }
 \end{lstlisting}
\end{frame}
%%%% Ab hier muss nur noch das Bild gewechselt werden...
\section{Fragen zum Übungsblatt?}
\begin{frame}{Fragen?}
\end{frame}



%Lacher zum Schluss
\begin{frame}
 \includegraphics[scale=1.5]{../pics/03_goodcomments}
 
 \tiny{Quelle: Google Bildersuche}
\end{frame}

\end{document}
