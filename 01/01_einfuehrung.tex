\documentclass{beamer}
\usepackage[utf8]{inputenc}

\title[Programmieren Tutorium]{1. Programmieren Tutorium:\\ Erste Schritte in Java}
\subtitle{Hello World / Datentypen}
\author{Konstantin Zangerle}

\usepackage{listings}
\usepackage{color}
\usetheme{Rochester}
\usepackage{biolinum}

\begin{document}


%title page
\begin{frame}
\titlepage
\end{frame}

%table of contents
\begin{frame}{Gliederung}
\tableofcontents
\end{frame}

\section{Vorstellung}
\subsection{Ich über mich}
\begin{frame}{Eigene Vorstellung}
\begin{itemize}
\item Konstantin Zangerle
\item 5. Semester Bachelor Informatik
\item info@konstantinzangerle.de
\item \dots
\end{itemize}
\end{frame}

\subsection{Freizeit}
\begin{frame}{Freizeitgedöns}
\begin{itemize}
\item öfter mal in der Fachschaft
\item steh auf Kaffee und Musik
\begin{itemize}
 \item E-Bass
 \item Ukulele
 \item Querflöte
 \item Gesang
\end{itemize}

\item \dots
\end{itemize}
\end{frame}

\subsection{Ihr seid an der Reihe!}
\begin{frame}{Eure Vorstellung}
\begin{itemize}
\item Name?
\item Studiengang? Informatik / InWi / \dots
\item Programmiererfahrung (mit C/Java/Python)
\item \dots
\end{itemize}
\end{frame}


\section{Organisatorisches}
\begin{frame}{ToDo Liste für euch!}
\begin{alertblock}{Gleich erledigen!}
\begin{itemize}
 \item Im ILIAS anmelden
 \item Im Praktomaten anmelden
 \item Einverständniserklärung im Briefkasten einwerfen
 \item VPN einrichten
 \item Java installieren
 \item Checkstyle installieren
 \item GIT installieren
\end{itemize}
\end{alertblock}
\end{frame}

\begin{frame}{Ablauf des Semesters}
\begin{alertblock}{}
\begin{itemize}
 \item Alle zwei Wochen Übungsblatt abgeben
 \item Am Ende mindestens 50\% der Punkte (wahrscheinlich 60P.)
 \item Im März 2 Abschlussaufgaben
  \begin{itemize}
        \item Nur die zählen zur Note!
        \item 2 Wochen versetzt
        \item 4 Wochen Bearbeitungszeit
       \end{itemize}
  \pause
  \item Alle Angaben wie immer ohne Gewähr
\end{itemize}
\end{alertblock}
\end{frame}


\section{Java}
\begin{frame}
 \begin{exampleblock}{Snelting}
  Java ist hier der Perverse!
 \end{exampleblock}

\end{frame}


\section{An die Arbeit}
\begin{frame}[fragile]{Erstes Programm}
\begin{exampleblock}{Hello World in Java}
\begin{lstlisting}[language=java]
public class HelloWorld {
  public static void main(String[] args) {
    System.out.println("Hello World");
  }
}
\end{lstlisting}
\end{exampleblock}
\end{frame}

\subsection{Interaktiver Teil}
\begin{frame}{Ihr passt an!}
\begin{exampleblock}{}
\begin{itemize}
\item Anderer Text
\pause
\item Text in eine Variable packen
\pause
\item Variablen? 
\end{itemize}
\end{exampleblock}
\end{frame}

\subsubsection{Übersicht Variablen}
\begin{frame}{Grundlegende Datentypen}
\begin{exampleblock}{Einmal zuordnen bitte!}
\begin{itemize}
\item int
\item String
\item double
\item float
\item long
\item char
\item \dots
\end{itemize}
\end{exampleblock}
\end{frame}

\subsection{Kleine Programmieraufgabe}
\begin{frame}{Ihr programmiert!}
\begin{exampleblock}{}
\begin{itemize}
\item Zwei Variablen (a,b)
\item Ausgabe:
\begin{itemize}
\item 1. Summand: a
\item 2. Summand: b
\item Summe: a+b
\end{itemize}
\end{itemize}
\end{exampleblock}
\end{frame}

\begin{frame}
 \includegraphics[scale=0.3]{pics/02_tutorslife} 
\end{frame}


\end{document}
