\documentclass[18pt]{beamer}
%% SLIDE FORMAT
\usepackage[utf8]{inputenc}
\usetheme{Rochester}

\title[Programmieren Tutorium]{2. Programmieren Tutorium:\\ Von Konstanten und Referenzen}
\subtitle{Datentypen / Referenzen / Datentypen und Schinken}
\author{Konstantin Zangerle \\ info@konstantinzangerle.de}

\usepackage{listings}
\usepackage{color}

\definecolor{mygreen}{rgb}{0,0.6,0}
\definecolor{mygray}{rgb}{0.5,0.5,0.5}
\definecolor{mymauve}{rgb}{0.58,0,0.82}

\lstset{ %
  backgroundcolor=\color{white},   % choose the background color
  basicstyle=\footnotesize,        % size of fonts used for the code
  breaklines=true,                 % automatic line breaking only at whitespace
  captionpos=b,                    % sets the caption-position to bottom
  commentstyle=\color{mygreen},    % comment style
  escapeinside={\%*}{*)},          % if you want to add LaTeX within your code
  keywordstyle=\color{blue},       % keyword style
  stringstyle=\color{mymauve},     % string literal style
}


\begin{document}

% change the following line to "ngerman" for German style date and logos
\selectlanguage{ngerman}

%title page
\begin{frame}
\titlepage
\end{frame}

%table of contents
\begin{frame}{Gliederung}
\tableofcontents
\end{frame}

\section{Confusing Code}
\begin{frame}[fragile]{Confusing Code}
\begin{exampleblock}{Was gibt dieser Code aus?}
\begin{lstlisting}[language=java]
public class FloatingPoint {
  public static void main(String[] args) {
    System.out.println(3* 0.1 == 0.3);
  }
}
\end{lstlisting}
\end{exampleblock}
\begin{itemize}
 \item true
 \item false
 \item Übersetzerfehler
 \item Laufzeitfehler
\end{itemize}
\end{frame}


\section{Wiederholung}
\subsection{Datentypen}
\begin{frame}{Datentypen}
\footnotesize
\begin{tabular}{|l|l|l|l|}
  \hline
  \textbf{Name} & \textbf{Verwendung} & \textbf{Anzahl Bits} & \textbf{Wertebereich} \\ \hline
 int & Ganzzahl & 32 & -2.147.483.648 \\
 & & & bis -2.147.483.647\\ \hline
 float & Gleitkommazahl & 32 & $-3.4028235 \cdot 10^{38}$ \\
 & & & bis $3.4028235 \cdot 10^{38}$\\\hline
 double & Gleitkommazahl & 64 & $-1.7976931348623157 \cdot 10^{308}$ \\ 
 & & & bis $1.7976931348623157 \cdot 10^{308}$ \\ \hline
 long & Ganzzahl & 64 & $-9.223.372.036.854.775.808$ \\ 
 & & & bis $9.223.372.036.854.775.807$ \\ \hline
 boolean & Wahrheitswerte & 1(?) & \{ true, false \} \\ \hline
 char & Utf-16 Zeichen & 16 &  0 bis 65535\\ \hline
 short  & Ganzzahl & 16 & -32.768 bis 32.767 \\ \hline
\end{tabular}
\end{frame}

\subsection{Konstruktoren}
\begin{frame}{Konstruktoren}
\begin{itemize}
\item werden beim Erstellen eines Objekts aufgerufen
\item public KLASSENNAME(PARAMETER)
\end{itemize}
\end{frame}

\begin{frame}[fragile]{Konstruktoren Beispiel}
\begin{exampleblock}{Schinken-Konstruktoren}
\begin{lstlisting}[language=java]
public class Schinken {
  int groesse;
  double gewicht;
  public Schinken() { // den gibts standardmaeszig
  }
  public Schinken(int groesse){ //den muss man explizit machen
    this.groesse = groesse;
  }
  public Schinken(int groesse, double gewicht) {
    this.groesse = groesse;
    this.gewicht = gewicht;
  }
}
\end{lstlisting}
\end{exampleblock}
\end{frame}

%% Operationen und Präzedenz
\section{Operatoren und Präzedenz}
\begin{frame}{Operatoren und Präzedenz}
 \begin{tabular}{|l|l|}
 \hline
  Präzedenz  & Operator \\
  \hline
  1 & +x, -x \\
  1 & !x \\
  2 & $x*y, x/y, x\%y $\\
  3 & x+y, x-y \\
  5 & $x<y, x <= y, x>y, x >= y$ \\
  6 & x == y, x!=y \\
  10 & x \&\& y \\
  11 & x $||$ y \\
   \hline
 \end{tabular}
\end{frame}


\begin{frame}[fragile]{Operatoren und Präzedenz Beispiel}
\begin{exampleblock}{Es kommt auf die Reihenfolge an?}
\begin{lstlisting}[language=java]
public class Schinken {
  [..]
  public int gibAnzahl(){
    if(Eater.getQuality() * Eater.length >= this.getWeight())
  }
  public int gibAnzahl(){
    if((Eater.getQuality() * Eater.length) >= this.getWeight())
  }
  
}
\end{lstlisting}
\end{exampleblock}
\end{frame}


%% Referenzen
\section{Referenzen}
\begin{frame}{Referenzen}
Man unterscheidet: \pause
\begin{itemize}
 \item Call by Reference 
  \begin{itemize}
    \item Funktionen bekommen eine Adresse übergeben
    \item Arbeiten mit den echten Objekten!
  \end{itemize}
  Call by Value
  \begin{itemize}
   \item Der übergebene Wert wird kopiert
   \item Die Funktion arbeitet mit Kopien!
  \end{itemize}

\end{itemize}
\end{frame}

\begin{frame}{Wie ist das in Java???}
\begin{itemize}
 \item Bei immutablen Datentypen: Call-by-Value (int, double, ....)
 \pause
 \item Bei mutablen Datentypen: Call-by-reference (Objekte aller Art)
 \pause
 \item Und bei Strings?
\end{itemize}

\end{frame}


\begin{frame}[fragile]{Und Strings?}
\begin{exampleblock}{Schinken-Konstruktoren}
\begin{lstlisting}[language=java] 
public class Test {
   public static void main(String args[]) {
      String Str1 = new String("This is really not immutable!!");
      String Str2 = Str1;
      String Str3 = new String("This is really not immutable!!");
      boolean retVal;

      retVal = Str1.equals( Str2 );
      System.out.println("Returned Value = " + retVal );

      retVal = Str1.equals( Str3 );
      System.out.println("Returned Value = " + retVal );
   }
}
\end{lstlisting}
\end{exampleblock}
\footnotesize Quelle: http://www.tutorialspoint.com/java/java\_string\_equals.htm
\end{frame}

\begin{frame}{Aufgabe}
 Schreibe ein Programm, bei dem ein Schinken an mindestens drei Personen gegeben wird. 
 Die Personen sollen dabei das Gewicht des Schinkens abschätzen und es gegebenfalls abändern.
 Jede Person hat außerdem einen Namen und einen typischen Satz.
\end{frame}


%% Konstantendeklaration und null
\section{Konstantendeklaration und null}
\begin{frame}{Konstantendeklaration und null}
\begin{itemize}
 \item final: Der Wert dieser Variable darf sich nicht ändern.
 \item null: Kein Objekt!
\end{itemize}
\pause
und nun zerstört null Programme...
\end{frame}


\section{Confusing Code}
\begin{frame}[fragile]{Confusing Code}
\begin{exampleblock}{Was gibt dieser Code aus?}
\begin{lstlisting}[language=java]
public class Test {
  public static void main(String[] args) {
    int i = (byte) + (char) - (int) + (long) - 1;
    System.out.println(i);
  }
}
\end{lstlisting}
\end{exampleblock}
\begin{itemize}
 \item 1
 \item -1
 \item 0
 \item 4294967296
 \item Übersetzerfehler
 \item Laufzeitfehler
\end{itemize}
\end{frame}


\section{Parameter}
\begin{frame}{Parameter}
Ein Parameter ist eine besondere Form von Variablen
Wir unterscheiden:
\begin{itemize}
 \item \textbf{Formaler Parameter}: \pause \\
  Der Name, der bei der Deklaration der Konstruktor Methode gewählt wurde.
  
 \item \textbf{Aktueller Parameter}:\pause \\
 Der Wert, der beim Aufruf für den formalen Parameter eingesetzt wird.
\end{itemize}
\end{frame}


\section{Konstruktoren}
\begin{frame}{Konstruktoren}
 Haben wir letzte Woche nocheinmal behandelt.
 Deswegen heute nur eine Aufgabe dazu!
 
 \textbf{Aufgabe:} Modelliere eine Klasse Quadrat, die als Konstruktoren die obere linke Ecke und die untere rechte Ecke annimmt.
 Ebenso soll es auch einen Konstruktor geben, der alle 4 Eckpunkte annimmt und auf Korrektheit überprüft, sowie einen Konstruktor
 der die obere linke Ecke sowie zusätzlich eine Länge annimmt.
\end{frame}

\section{enum}
\begin{frame}{enum}
 Oft will man in einem Programm eine Auswahl bereitstellen. Es soll also eine Klasse geben,
 die es erlaubt einen begrenzten Wertebereich anzugeben.
 Bsp.: Für ein Programm, dass mit Monaten arbeitet ist es blöd, integer als Parameter zu verwenden. Das erfordert immer eine
 zusätzliche Überprüfung! \\
 \pause
 \textbf{Lösung: } ENUM!
\end{frame}
\begin{frame}[fragile]{enum - Beispiel}
 \begin{lstlisting}
  public enum Months {
    JAN, FEB, MAR, APR, MAI, JUN, JUL, SEP, OCT, NOV, DEZ
  }
 \end{lstlisting}
\end{frame}

\begin{frame}{enum - Aufgabe}
 Sicherlich bekannt sind die Lieder: 
 \begin{itemize}
  \item Manic Monday
  \item Ruby Tuesday
  \item Wednesday's Song
  \item Thursday
  \item Friday I'm in love
  \item Saturday night fever
  \item Sunday Morning
 \end{itemize}
 Schreibt ein Programm, dass per Eingabe eines enum Typs mir den richtigen Song ausgiebt. 
 
 (Bonuspunkte für den richtigen Interpreten)
\end{frame}

\begin{frame}{enum - Aufgabe}
 Sicherlich bekannt sind die Lieder: 
 \begin{itemize}
  \item Manic Monday – Bangles
  \item Ruby Tuesday – Rolling Stones
  \item Wednesday's Song – John Frusciante
  \item Thursday – Pet Shop Boys
  \item Friday I'm in love – The Cure
  \item Saturday night fever – Bee Gees
  \item Sunday Morning – No Doubt
 \end{itemize}
 Schreibt ein Programm, dass per Eingabe eines enum Typs mir den richtigen Song ausgiebt. 
 
 (Bonuspunkte für den richtigen Interpreten)
\end{frame}

\section{Methoden}
\subsection{Methodensignatur}
\begin{frame}[fragile]{Methoden}
 \begin{lstlisting}
  public int convertToFahrenheit(int celsius) {
    return (celsius * 9) / 5 + 32;
  }
 \end{lstlisting}
 ist Methode der Klasse Temperatur. Die Methodensignatur ist der Fingerabdruck der Methode für den Compiler.
 
 Frage: Aus was besteht dieser Fingerabdruck?
\end{frame}

\begin{frame}{Methodensignatur}
Antwort:
 \begin{itemize}
  \item Namen der Methode
  \item Anzahl der Parameter
  \item Reihenfolge der Parameter
  \item Typen der Parameter
  \item Rückgabetyp der Methode
 \end{itemize}

\end{frame}

\subsection{getter und setter}
\begin{frame}[fragile]{getter und setter}
Will man den Zugriff auf die Attribute einer Klasse einschränken will aber trotzdem das Auslesen ermöglichen so bietet man den Zugriff über eine Getter-Methode an.
\begin{lstlisting}
 public class Atomwaffe {
  private double temperatur; // das zu verraten schadet nichts?
  private double uranmenge; //veraendern 
  private boolean verraten = true;
  public void setTemperatur(double temp) { //erlaube Aenderung der 
  //Temperatur
    if (temp < 200 && temp > 0) { //aber nur in Maszen
      temperatur = temp;
    }
  public double getUranmenge() {
    if (verraten == true) {
      return uranmenge;
    } else {
      return Double.NaN;
    }
  }
 }
\end{lstlisting}
\end{frame}



\subsection{Überladen von Methoden}
\begin{frame}{Überladen von Methoden}
 Methoden können den gleichen Namen haben – solange die Signatur unterschiedlich ist, weiß der Compiler/Interpreter was er nehmen soll.
 Ihr habt heute schon Methoden überladen!? \\ \pause
 \textbf{Genau! Den Konstruktor}
\end{frame}

\begin{frame}[fragile]{Überladen – noch ein Beispiel}
 \begin{lstlisting}
public class Ueberladen {
    public int max(int i1, int i2) { 
        return (i1 > i2) ? i1 : i2; 
    }
   
    public float max(float f1, float f2) { 
        return (f1 > f2) ? f1 : f2; 
    }
    
    public double max(double d1, double d2) { 
        return (d1 > d2) ? d1 : d2; 
    }
} 
 \end{lstlisting}
\end{frame}

\section{static – Was ist das?}
\begin{frame}{static was ist das?}
 static zeigt an dass etwas – Methode oder Attribut – nicht zum Objekt sondern zur Klasse, d.h. für alle Instanzen der Klasse genau einmal existiert.
 Was bedeutet das für uns?
\end{frame}

\begin{frame}{static – die Auswirkungen}
\begin{itemize}
 \item eine static Funktion kann keine nicht-static Attribute benutzen
 \item eine static Funktion kann keine nicht-static Funktion benutzen
 \item Umgekehrt funktioniert aber beides!
 \item Kann für Konstanten verwendet werden
\end{itemize}
 

\end{frame}


\begin{frame}[fragile]{Konkrete for – macht ihr Selbst!}
 \begin{lstlisting}
 System.out.println("Einmal");
 int i = 10;
 while(i > 0) {
  System.out.println("Und nocheinmal...");
  i--;
 }
 \end{lstlisting}
 Aufgabe: Wandle obige WHILE-Schleife in eine FOR-Schleife um!
\end{frame}

\begin{frame}[fragile]{Sollte aber ungefähr so aussehen!}
 \begin{lstlisting}
 System.out.println("Einmal");
 for (int i = 10; i > 0; i--) {
   System.out.println("Und nocheinmal...");
 }
 \end{lstlisting}
\end{frame}
%%%% Ab hier muss nur noch das Bild gewechselt werden...
\section{Fragen zum Übungsblatt?}
\begin{frame}{Fragen?}
\end{frame}



%Lacher zum Schluss
\begin{frame}
 \includegraphics[scale=1.5]{../pics/03_goodcomments}
 
 \tiny{Quelle: Google Bildersuche}
\end{frame}

\end{document}
