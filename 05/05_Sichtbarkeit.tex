%% LaTeX-Beamer template for KIT design
%% by Erik Burger, Christian Hammer
%% title picture by Klaus Krogmann
%%
%% version 2.1
%%
%% mostly compatible to KIT corporate design v2.0
%% http://intranet.kit.edu/gestaltungsrichtlinien.php
%%
%% Problems, bugs and comments to
%% burger@kit.edu

\documentclass[18pt]{beamer}
%% SLIDE FORMAT
\usepackage[utf8]{inputenc}
\usepackage{algpseudocode}
% use 'beamerthemekit' for standard 4:3 ratio
% for widescreen slides (16:9), use 'beamerthemekitwide'

\usepackage{templates/beamerthemekit}
% \usepackage{templates/beamerthemekitwide}


%% TikZ INTEGRATION

% use these packages for PCM symbols and UML classes
% \usepackage{templates/tikzkit}
% \usepackage{templates/tikzuml}

% the presentation starts here

\title[Programmieren Tutorium]{5. Programmieren Tutorium:\texorpdfstring{\\}{} Warum Objektorientierung?}
\subtitle{Sichtbarkeiten}
\author{Konstantin Zangerle \texorpdfstring{\\}{} info@konstantinzangerle.de}
\date{25. November}
\institute{Chair for Software Design and Quality}

\usepackage{listings}
\usepackage{color}

\definecolor{mygreen}{rgb}{0,0.6,0}
\definecolor{mygray}{rgb}{0.5,0.5,0.5}
\definecolor{mymauve}{rgb}{0.58,0,0.82}

\lstset{ %
  backgroundcolor=\color{white},   % choose the background color
  basicstyle=\footnotesize,        % size of fonts used for the code
  breaklines=true,                 % automatic line breaking only at whitespace
  captionpos=b,                    % sets the caption-position to bottom
  commentstyle=\color{mygreen},    % comment style
  escapeinside={\%*}{*)},          % if you want to add LaTeX within your code
  keywordstyle=\color{blue},       % keyword style
  stringstyle=\color{mymauve},     % string literal style
  showstringspaces=false,
  language=Java
}
\beamersetuncovermixins{\opaqueness<1>{0}}{\opaqueness<2->{0}} %Dont show things after pause 
% Bibliography

\usepackage[citestyle=authoryear,bibstyle=numeric,hyperref,backend=biber]{biblatex}
\addbibresource{templates/example.bib}
\bibhang1em


\begin{document}

% change the following line to "ngerman" for German style date and logos
\selectlanguage{ngerman}

%title page
\begin{frame}
\titlepage
\end{frame}

%table of contents
\begin{frame}{Gliederung}
\tableofcontents
\end{frame}

\section{Organisatorisches} % 0-3
\begin{frame}{Euer Tutorium – eure Wahl!}
  \begin{itemize}
   \item Jede Woche Tutorium
   \item Nur alle zwei Wochen Aufgabenblätter
   \item Möglichkeit 1: Ein Stoff-Tut, Ein Übe-Tut
   \item Möglichkeit 2: Stoff verteilt auf zwei Tuts, Übungen zwischendurch
  \end{itemize}
\end{frame}

\section{Arrays} % 4-7
\begin{frame}[fragile]{Arrays}
\begin{lstlisting}
/* returns max {a,b,c,d} */
public static int max4(int a, int b, int c, int d) { } 

/* returns max {a,b,c,d,e} */
public static int max5(int a, int b, int c, int d, int e) { } 


/* any way to do this better? */
\end{lstlisting}
\end{frame}

\begin{frame}[fragile]{Arrays}
 \begin{lstlisting}
    public static int max(int[] a) {
        int max = Integer.MIN_VALUE;
        for (int k : a) {
            if (k >= max) {
                max = k;
            }
        }
        return max;
    }
 \end{lstlisting}
\end{frame}

\begin{frame}{Arrays – Was muss man beachten}
 \begin{itemize}
  \item Arrays sollen Daten zusammenfassen, die zusammengehören
  \item Auch wenns geht, so ein Array ist blöd [Handelsklasse, Gewicht, Volumen, Gewicht]
  \item Eignen sich für feste Datengrößen (5x int; double-Feld 5x3; ...)
 \end{itemize}

\end{frame}


\section{Typkonvertierung}
\begin{frame}[fragile]{Typkonvertierung – funktioniert häufig}
Bei Basisdatentypen den gewünschten Datentyp in Klammern schreiben.
 \begin{lstlisting}
  int a = (int) 5.3;
  System.out.println(a);
 \end{lstlisting}
 \pause
Java übernimmt das umwandeln. Manchmal muss man noch nichteinmal das machen.
 \begin{lstlisting}
  double a = 5;
  System.out.println(a);
 \end{lstlisting}
\end{frame}

\section{Datenkapselung}
\begin{frame}{Datenkapselung}
 Eines der grundlegenden Prinzipen der OOP.
 Daten sollen dahin, wo sie hingehören (und nur da!).
  Wenn möglich, geschieht dies nach dem \textbf{Geheimnisprinzip}. Jedes
  Objekt, jede Klasse ändert nur die Dinge, die ihr auch selbst gehören.
\end{frame}

\subsection{Sichtbarkeit}
\begin{frame}[fragile]{Sichtbarkeit}
 Um Programmierer dazu zu zwingen, die Datenkapselung einzuhalten,
 benutzen wir Sichtbarkeiten. Ist ein Attribut oder eine Methode als 
 \verb|private| gekennzeichent, so ist ein Zugriff nur innerhalb
 derselben Klasse möglich.
 Es gibt (in Java) drei Sichtbarkeiten:
 \begin{itemize}
  \item public
  \item protected
  \item private
 \end{itemize}
\end{frame}

\subsection{Packages}

\section{Mehr Stoff im Wiki!}
\subsection{Checkstyle}
\begin{frame}{Checkstyle}
 Insert interactive demonstration here
\end{frame}
\subsection{Eclipse}
\begin{frame}{Eclipse}
 Insert interactive demonstration here
\end{frame}
\section{Graphentheorie}
\begin{frame}{Was ist ein Graph?}
 Schwierige Frage.
 \begin{itemize}
  \item Eine Zeichnung mit Knoten und Kanten \pause
  \item Eine Operation auf einer Menge. \pause
  \item Eine Veranschaulichung einer Relation \pause
  \item Was ist eine Relation?
 \end{itemize}
\end{frame}

\subsection{Darstellung als Arrays}
\begin{frame}{Graphen als Arrays}
Kommt genauer im zweiten Semester. Hier dürft ihr das machen.
Ihr benötigt zwei Arrays. Eine speichert, wie viele Kanten von einer
Kante ausgehen. Die andere speichert die Ziele der Kanten.

\textbf{Aufgabe:} Entwerfe eine Klasse Graph, die einen Graphen speichern
kann. Die Knoten sollen hierbei von 0 durchnummeriert werden.
\end{frame}

\section{Matrizen}
\begin{frame}{Matrizen}
 Was ist eine Matrix? \pause
 
 Grob ist eine Matrix eine Tabelle. 
 Tabellen können als Arrays gespeichert werden. 
 
 \pause
 Verwende zweidimensionale Arrays!
\end{frame}
%%%% Ab hier muss nur noch das Bild gewechselt werden...
\section{Fragen zum Übungsblatt?}
\begin{frame}{Fragen?}
\end{frame}



%Lacher zum Schluss
\begin{frame}
 \includegraphics[scale=0.25]{05_afraidtodebug}
 
 \tiny{Quelle: 9gag.com}
\end{frame}


\end{document}
%%%%%%%%%%%%%%%%%%%%%%%%%
%Bausteine
%nice to have code
%%%%%%%%%%%%%%%%%%%%%%%%%%



%%%%% Bausteine Folie mit Java-Code
%\begin{frame}[fragile]{bla}
%\begin{exampleblock}{bla}
%\begin{lstlisting}[language=java]
%\end{lstlisting}
%\end{exampleblock}
%\end{frame}
