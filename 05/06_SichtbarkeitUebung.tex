%% LaTeX-Beamer template for KIT design
%% by Erik Burger, Christian Hammer
%% title picture by Klaus Krogmann
%%
%% version 2.1
%%
%% mostly compatible to KIT corporate design v2.0
%% http://intranet.kit.edu/gestaltungsrichtlinien.php
%%
%% Problems, bugs and comments to
%% burger@kit.edu

\documentclass[18pt]{beamer}
%% SLIDE FORMAT
\usepackage[utf8]{inputenc}
\usepackage{algpseudocode}
% use 'beamerthemekit' for standard 4:3 ratio
% for widescreen slides (16:9), use 'beamerthemekitwide'

\usepackage{templates/beamerthemekit}
% \usepackage{templates/beamerthemekitwide}


%% TikZ INTEGRATION

% use these packages for PCM symbols and UML classes
% \usepackage{templates/tikzkit}
% \usepackage{templates/tikzuml}

% the presentation starts here

\title[Programmieren Tutorium]{5. Programmieren Tutorium:\texorpdfstring{\\}{} Warum Objektorientierung?}
\subtitle{Sichtbarkeiten}
\author{Konstantin Zangerle \texorpdfstring{\\}{} info@konstantinzangerle.de}
\date{2. Dezember}
\institute{Chair for Software Design and Quality}

\usepackage{listings}
\usepackage{color}

\definecolor{mygreen}{rgb}{0,0.6,0}
\definecolor{mygray}{rgb}{0.5,0.5,0.5}
\definecolor{mymauve}{rgb}{0.58,0,0.82}

\lstset{ %
  backgroundcolor=\color{white},   % choose the background color
  basicstyle=\footnotesize,        % size of fonts used for the code
  breaklines=true,                 % automatic line breaking only at whitespace
  captionpos=b,                    % sets the caption-position to bottom
  commentstyle=\color{mygreen},    % comment style
  escapeinside={\%*}{*)},          % if you want to add LaTeX within your code
  keywordstyle=\color{blue},       % keyword style
  stringstyle=\color{mymauve},     % string literal style
  showstringspaces=false,
  language=Java
}
\beamersetuncovermixins{\opaqueness<1>{0}}{\opaqueness<2->{0}} %Dont show things after pause 
% Bibliography

\usepackage[citestyle=authoryear,bibstyle=numeric,hyperref,backend=biber]{biblatex}
\addbibresource{templates/example.bib}
\bibhang1em


\begin{document}

% change the following line to "ngerman" for German style date and logos
\selectlanguage{ngerman}

%title page
\begin{frame}
\titlepage
\end{frame}

%table of contents
\begin{frame}{Gliederung}
\tableofcontents
\end{frame}
\section{Übungsblatt}
\subsection{Statistik}
\begin{frame}{Statistik}
\begin{itemize}
 \item Teil A: 5.16666666667
 \item Teil B: 5.47222222222 	
 \item Teil C: 3.02777777778 	
 \item Teil D: 1.95238095238 	
 \item Ins: 15,06 (von 22 ungefähr 68\%) 
\end{itemize}
So weit also gut :)
\end{frame}

\subsection{Strings}
\begin{frame}[fragile]{Strings}
 \includegraphics[scale=0.5]{stringtime}
 \begin{itemize}
  \item Strings sind keine primitiven Datentypen
  \item \verb|String test = "a";| test ist ein ``normales'' String-Objekt 
  \item \verb|new String| zwingt Java dazu ein neues Objekt anzulegen, wird nicht ``wegoptimiert''
 \end{itemize}
\end{frame}

\subsection{Bijektion}
\begin{frame}{Bijektion}
 War schwer.
 
 \includegraphics[scale=0.5]{Dozent} \pause
 
 Hat niemand.
\end{frame}

\begin{frame}[fragile]{Bijektion}
 Was wurde erwartet? \pause
 In einem der beiden Klassen sowas:
 \begin{lstlisting}
  public class Lecturer {
    [...]
    Lecture job;
    
    public Lecturer([...], Lecturer job) {
      job.setLecturer(this);
      [...]
    }
  }
 \end{lstlisting}

\end{frame}

\section{Packages}
\subsection{Definitionen}
\begin{frame}{Packages}
 Sind die Module in Java. \pause
 
 \begin{exampleblock}{Definition: Modul}
  Ein Modul (neutrum, das Modul[1]) ist eine abgeschlossene funktionale Einheit einer Software, 
  bestehend aus einer Folge von Verarbeitungsschritten und Datenstrukturen. Inhalt eines Moduls ist
  häufig eine wiederkehrende Berechnung oder Bearbeitung von Daten, die mehrfach durchgeführt 
  werden muss. Das Modul führt eine Reihe von Verarbeitungsschritten durch, liefert bei der 
  Rückkehr an das aufrufende Programm Daten als Ergebnis zurück.
  
  \tiny{Quelle: http://de.wikipedia.org/wiki/Modul\_\%28Software\%29 }
 \end{exampleblock}

\end{frame}

\begin{frame}{Packages}
 Sind die Module in Java. \pause
 
 \begin{exampleblock}{Definition: Modul (SWT)}
 Ein Modul ist eine Menge von Programmelementen, die nach dem Geheimnisprinzip gemeinsam entworfen und geändert werden.

 \end{exampleblock}

\end{frame}

\begin{frame}{Packages}
\tiny
A package is a namespace that organizes a set of related classes and interfaces. 
Conceptually you can think of packages as being similar to different folders on your computer. 
You might keep HTML pages in one folder, images in another, and scripts or applications in yet another. 
Because software written in the Java programming language can be composed of hundreds or thousands of 
individual classes, it makes sense to keep things organized by placing related classes and interfaces into packages.

The Java platform provides an enormous class library (a set of packages) suitable for use in your own 
applications. This library is known as the ``Application Programming Interface'', or ``API'' for short. 
Its packages represent the tasks most commonly associated with general-purpose programming. For example, 
a String object contains state and behavior for character strings; a File object allows a programmer to easily 
create, delete, inspect, compare, or modify a file on the filesystem; a Socket object allows for the creation
and use of network sockets; various GUI objects control buttons and checkboxes and anything else related to 
graphical user interfaces. There are literally thousands of classes to choose from. This allows you, the programmer, 
to focus on the design of your particular application, rather than the infrastructure required to make it work.

\emph{The Java Platform API Specification contains the complete listing for all packages, interfaces, classes, fields, 
and methods supplied by the Java SE platform. Load the page in your browser and bookmark it. As a programmer, 
it will become your single most important piece of reference documentation.}

\end{frame}

\begin{frame}{Packages}
\scriptsize
A package is a namespace that organizes a set of related classes and interfaces. 
Conceptually you can think of packages as being similar to different folders on your computer. 
You might keep HTML pages in one folder, images in another, and scripts or applications in yet another. 
Because software written in the Java programming language can be composed of hundreds or thousands of 
individual classes, it makes sense to keep things organized by placing related classes and interfaces into packages.

The Java platform provides an enormous class library (a set of packages) suitable for use in your own 
applications. This library is known as the ``Application Programming Interface'', or ``API'' for short. 
Its packages represent the tasks most commonly associated with general-purpose programming. For example, 
a String object contains state and behavior for character strings; a File object allows a programmer to easily 
create, delete, inspect, compare, or modify a file on the filesystem; a Socket object allows for the creation
and use of network sockets; various GUI objects control buttons and checkboxes and anything else related to 
graphical user interfaces. There are literally thousands of classes to choose from. This allows you, the programmer, 
to focus on the design of your particular application, rather than the infrastructure required to make it work.

\emph{The Java Platform API Specification contains the complete listing for all packages, interfaces, classes, fields, 
and methods supplied by the Java SE platform. Load the page in your browser and bookmark it. As a programmer, 
it will become your single most important piece of reference documentation.}

\end{frame}

\begin{frame}{Packages}
\begin{alertblock}{Wichtig}
\Large
\emph{The Java Platform API Specification contains the complete listing for all packages, interfaces, classes, fields, 
and methods supplied by the Java SE platform. Load the page in your browser and bookmark it. As a programmer, 
it will become your single most important piece of reference documentation.}
 
\end{alertblock}

\end{frame}

%%%% Ab hier muss nur noch das Bild gewechselt werden...
\section{Fragen zum Übungsblatt?}
\begin{frame}{Fragen?}
\end{frame}



%Lacher zum Schluss
\begin{frame}{Approximate?}
 \includegraphics[scale=0.6]{06_approximate}
 
 \tiny{Quelle: xkcd.com}
\end{frame}


\end{document}
%%%%%%%%%%%%%%%%%%%%%%%%%
%Bausteine
%nice to have code
%%%%%%%%%%%%%%%%%%%%%%%%%%

sv

%%%%% Bausteine Folie mit Java-Code
%\begin{frame}[fragile]{bla}
%\begin{exampleblock}{bla}
%\begin{lstlisting}[language=java]
%\end{lstlisting}
%\end{exampleblock}
%\end{frame}
