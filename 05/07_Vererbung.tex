%% LaTeX-Beamer template for KIT design
%% by Erik Burger, Christian Hammer
%% title picture by Klaus Krogmann
%%
%% version 2.1
%%
%% mostly compatible to KIT corporate design v2.0
%% http://intranet.kit.edu/gestaltungsrichtlinien.php
%%
%% Problems, bugs and comments to
%% burger@kit.edu

\documentclass[18pt]{beamer}
%% SLIDE FORMAT
\usepackage[utf8]{inputenc}
\usepackage{algpseudocode}
\usepackage{amssymb}
% use 'beamerthemekit' for standard 4:3 ratio
% for widescreen slides (16:9), use 'beamerthemekitwide'

\usepackage{templates/beamerthemekit}
% \usepackage{templates/beamerthemekitwide}


%% TikZ INTEGRATION

% use these packages for PCM symbols and UML classes
% \usepackage{templates/tikzkit}
% \usepackage{templates/tikzuml}

% the presentation starts here

\title[Programmieren Tutorium]{7. Programmieren Tutorium:\texorpdfstring{\\}{}Vererbung?}
\subtitle{und vieles was dazu gehört}
\author{Konstantin Zangerle \texorpdfstring{\\}{} info@konstantinzangerle.de}
\date{9. Dezember}
\institute{Chair for Software Design and Quality}

\usepackage{listings}
\usepackage{color}

\definecolor{mygreen}{rgb}{0,0.6,0}
\definecolor{mygray}{rgb}{0.5,0.5,0.5}
\definecolor{mymauve}{rgb}{0.58,0,0.82}

\lstset{ %
  backgroundcolor=\color{white},   % choose the background color
  basicstyle=\footnotesize,        % size of fonts used for the code
  breaklines=true,                 % automatic line breaking only at whitespace
  captionpos=b,                    % sets the caption-position to bottom
  commentstyle=\color{mygreen},    % comment style
  escapeinside={\%*}{*)},          % if you want to add LaTeX within your code
  keywordstyle=\color{blue},       % keyword style
  stringstyle=\color{mymauve},     % string literal style
  showstringspaces=false,
  language=Java
}
\beamersetuncovermixins{\opaqueness<1>{0}}{\opaqueness<2->{0}} %Dont show things after pause 
% Bibliography

\usepackage[citestyle=authoryear,bibstyle=numeric,hyperref,backend=biber]{biblatex}
\addbibresource{templates/example.bib}
\bibhang1em


\begin{document}
% change the following line to "ngerman" for German style date and logos
\selectlanguage{ngerman}

%title page
\begin{frame}
\titlepage
\end{frame}

%table of contents
\begin{frame}{Gliederung}
\tableofcontents
\end{frame}
\section{Rückblick}
\begin{frame}{Rückblick}
 \begin{itemize}
  \item Klassen
  \item Objekte
  \item Attribute
  \item Variablen
  \item Datentypen
  \item Konstruktoren
  \item Methoden
  \item Kontrollstrukturen
  \item Arrays
  \item Checkstyle
  \item Präzedenz
  \item String
  \item Referenzen
 \end{itemize}
\end{frame}

\section{Heute + nächste Woche}
\begin{frame}[fragile]{Ausblick}
\begin{itemize}
 \item Vererbung
 \item Dynamische Bindung
 \item Überschreibung von Attributen und Methoden
 \item Konstruktoren mit \verb|super|
 \item Typumwandlungen
 \item Klasse \verb|Object| und Java-Klassenhierarchie
 \item Abstrakte Klassen
 \item Inhaltliche Gleichheit
 \item \verb|final|
 \item Javadoc
 \item Verkette Listen und Iteratoren
 \item Queue, Stack, Priority Queue
 \item Schnittstellen
 \item Generische Klassen und Schnittstellen
  \item \verb|instanceof|
\end{itemize}
\end{frame}

\begin{frame}[fragile]{Heute}
\begin{itemize}
 \item Vererbung
 \item Überschreibung von Attributen und Methoden
 \item Generische Klassen und Schnittstellen
 \item Javadoc
 \item Konstruktoren mit \verb|super|
 \item Schnittstellen 
\end{itemize}
\end{frame}

\section{Vererbung}
\subsection{Einführung}
\begin{frame}{Vererbung}
 \includegraphics[scale=0.7]{07_AutoVererbung.png}
 \begin{itemize}
 \item Ein PKW ist ein Fahrzeug, ein Zweirad ist ein Fahrzeug, 
  \item Eine Limousine ist ein Fahrzeug, aber nicht jedes Fahrzeug ist eine Limousine
  \item Jedes Cabrio kann das Dach öffnen bzw. schließen, bei PKWs macht das (meistens) keinen Sinn.
  \item \ldots
 \end{itemize}
\end{frame}

\begin{frame}{Genauer: Vererbung}
 \begin{itemize}
  \item Vererbung ist eine ``ist''-Beziehung
  \item Alle Eigenschaften (Attribute) werden übernommen!
  \item Alle Methoden werden übernommen.
  \item Kinder haben aber auch Freiheiten diese zu ändern!
 \end{itemize}
\end{frame}


\subsection{In Java?!}
\begin{frame}[fragile]{In Java!}
 \begin{enumerate}
  \item Überlege welche Klassen du benötigst
  \item Überlege welche sinnvollen Vererbungsbeziehungen du aufstellen kannst!
  \item Erstelle jede Klasse ``normal''.
  \item Benutze \verb|public class X extends Schinken {|
 \end{enumerate}
\end{frame}

\subsection{Aufgabe zur Vererbung}
\begin{frame}{Übung!}
 \includegraphics[scale=0.7]{07_AutoVererbung.png}
 \begin{itemize}
  \item Modelliere den Abschnitt PKW, Cabrio, Limousine, Coupé
  \item Füge sinnvolle Attribute in sinnvollen Klassen hinzu!
  \item Füge sinnvolle Methoden in sinnvollen Klassen hinzu!
  \item Benutze Vererbung!
 \end{itemize}

\end{frame}

\begin{frame}[fragile]{Heute}
\begin{itemize}
 \item Vererbung \checkmark
 \item Überschreibung Methoden und Überlagerung von Attributen
 \item Generische Klassen und Schnittstellen
 \item Javadoc
 \item Konstruktoren mit \verb|super|
 \item Schnittstellen 
\end{itemize}
\end{frame}

\section{Methoden und Methoden in Vererbung}
\begin{frame}[fragile]{Überschreibung von Methoden}
Betrachte die Klassen
\begin{lstlisting}
public class Motor {
  public void starten() { ... }
}

public class DieselMotor extends Motor {
  @Override
  public void starten() { ... }
} 
\end{lstlisting}
\pause
\begin{itemize}
 \item Warum ist diese Vererbung sinnvoll? \pause
 \item Warum sollte man die starten Methode überschreiben? \pause
 \item Warum steht hier @Override? \pause
 \item Warum liegt hier überhaupt Stroh rum?
\end{itemize}
\end{frame}

\begin{frame}{Aufgabe}
 \begin{itemize}
  \item Jeder (funktionierende) PKW hat ein Motor? Oder?
  \item Wenn dem so ist, füge dies als Attribut hinzu.
  \item Schreibe einen Konstruktor, der ein Motorobjekt annimmt und diesen an die super-Klasse PKW weitergibt.
 \end{itemize}
\end{frame}

\begin{frame}[fragile]{Überschatten von Attributen}
 \begin{lstlisting}
  class SuperBoaster {
  int nr = 1;
  void boast()  {
    System.out.println( "Ich bin die Nummber " + nr );
  }
}
public class SubBoaster extends SuperBoaster {
  int nr = 2;
  @Override 
  void boast()  {
    super.boast();                   // Ich bin die Nummber 1
    System.out.println( super.nr );  // 1
    System.out.println( nr );        // 2
  }

  public static void main( String[] args )  {
    new SubBoaster().boast();
  }
} //Geklaut von: Java ist auch eine Insel 5.11.5
 \end{lstlisting}
\end{frame}

\begin{frame}[fragile]{Heute}
\begin{itemize}
 \item Vererbung \checkmark
 \item Überschreibung Methoden und Überlagerung von Attributen \checkmark
 \item Generische Klassen und Schnittstellen
 \item Javadoc
 \item Konstruktoren mit \verb|super|

 \item Schnittstellen 
\end{itemize}
\end{frame}

\begin{frame}{Konstruktoren und super}
 \begin{alertblock}{Ein wichtiger Unterschied}
  In Konstruktoren muss der super() Konstruktor als erstes aufgerufen werden,
  oder gar nicht!
 \end{alertblock}
\end{frame}


\begin{frame}[fragile]{Generische Klassen}
 \begin{lstlisting}
  ArrayList<String> liste = new ArrayList<String>(); //oder...
  Liste liste = new ArrayList<String>();
 \end{lstlisting}
\end{frame}

\begin{frame}[fragile]{Heute}
\begin{itemize}
 \item Vererbung \checkmark
 \item Überschreibung Methoden und Überlagerung von Attributen \checkmark
 \item Generische Klassen und Schnittstellen \checkmark
 \item Javadoc
 \item Konstruktoren mit \verb|super| \checkmark
 \item Schnittstellen
\end{itemize}
\end{frame}


\begin{frame}[fragile]{Schnittstellen}
\begin{lstlisting}
 interface Motor {
  public void starten();
  public int getLeistung();
 }
 \end{lstlisting}
 \begin{itemize}
  \item Interfaces stellen eine Schablone dar.
  \item Interfaces geben die Gewissheit, das Klassen, die diese implementieren, Funktionen besitzen.
  \item Interfaces haben keine Attribute.
 \end{itemize}

\end{frame}

\begin{frame}[fragile]{Javadoc}
 Javadoc sind Kommentare mit folgendem Schema:
 \begin{lstlisting}
  /**
 * Kurzbeschreibung der Funktionalitaet in einem Satz.
 * Weitere Details folgen darauf.
 * 
 * Beschreibung von Parametern, Rueckgabewerten und co.
 * geschieht durch spezielle Tags.
 */
 \end{lstlisting}
Außerdem können ``Tags'' verwendet werden, bspw. @author, @version, @param, @return, @throws
Siehe dazu den Eintrag im Programmieren Wiki
\scriptsize
\verb|https://ilias.studium.kit.edu/goto.php?target=wiki_349162_Javadoc|
\end{frame}

\begin{frame}[fragile]{Heute}
\begin{itemize}
 \item Vererbung \checkmark
 \item Überschreibung Methoden und Überlagerung von Attributen \checkmark
 \item Generische Klassen und Schnittstellen \checkmark
 \item Javadoc \checkmark
 \item Konstruktoren mit \verb|super| \checkmark
 \item Schnittstellen \checkmark
\end{itemize}
\end{frame}


%%%% Ab hier muss nur noch das Bild gewechselt werden...
\section{Fragen zum Übungsblatt?}
\begin{frame}{Fragen?}
\end{frame}



%Lacher zum Schluss
\begin{frame}{Where is my time? Oh, I'm a tutor.}
 \includegraphics[scale=0.25]{07_time}
 
 \tiny{Quelle: xkcd.com}
\end{frame}


\end{document}
%%%%%%%%%%%%%%%%%%%%%%%%%
%Bausteine
%nice to have code
%%%%%%%%%%%%%%%%%%%%%%%%%%


%%%%% Bausteine Folie mit Java-Code
%\begin{frame}[fragile]{bla}
%\begin{exampleblock}{bla}
%\begin{lstlisting}[language=java]
%\end{lstlisting}
%\end{exampleblock}
%\end{frame}
