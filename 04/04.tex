\documentclass[18pt]{beamer}
%% SLIDE FORMAT
\usepackage[utf8]{inputenc}
\usepackage{algpseudocode}

\title[Programmieren Tutorium]{4. Programmieren Tutorium:\\ Übungen}
\subtitle{Kontrolle ist gut – Kontrolle ist besser}
\author{Konstantin Zangerle \\ info@konstantinzangerle.de}
\date{\today}

\usepackage{listings}
\usepackage{color}

\begin{document}

% change the following line to "ngerman" for German style date and logos
\selectlanguage{ngerman}

%title page
\begin{frame}
\titlepage
\end{frame}

%table of contents
\begin{frame}{Gliederung}
\tableofcontents
\end{frame}

\section{Übungsaufgaben}
\subsection{Binäre Suche}
\begin{frame}{Aufgabe: Binäre Suche}
Ich denke mir eine Zahl zwischen Eins und Hundert. Schreibe ein Programm, 
dass für jede mögliche Zahl $n \in \mathbb{N}, n <101$ n mit möglichst 
wenigen Fragen der Form ``Kleiner, Größer, Gleich [1-3]'' errät.
\end{frame}



\section{Arrays} % 4-7
\begin{frame}[fragile]{Arrays}
\begin{lstlisting}[language=java]
/* returns max {a,b,c,d} */
public static int max4(int a, int b, int c, int d) { } 

/* returns max {a,b,c,d,e} */
public static int max5(int a, int b, int c, int d, int e) { } 


/* any way to do this better? */
\end{lstlisting}
\end{frame}

\begin{frame}[fragile]{Arrays}
 \begin{lstlisting}[language=java]
    public static int max(int[] a) {
        int max = Integer.MIN_VALUE;
        for (int k : a) {
            if (k >= max) {
                max = k;
            }
        }
        return max;
    }
 \end{lstlisting}
\end{frame}

\begin{frame}{Arrays – Was muss man beachten}
 \begin{itemize}
  \item Arrays sollen Daten zusammenfassen, die zusammengehören
  \item Auch wenns geht, so ein Array ist blöd [Handelsklasse, Gewicht, Volumen, Gewicht]
  \item Eignen sich für feste Datengrößen (5x int; double-Feld 5x3; ...)
 \end{itemize}

\end{frame}


\section{Typkonvertierung}
\begin{frame}[fragile]{Typkonvertierung – funktioniert häufig}
Bei Basisdatentypen den gewünschten Datentyp in Klammern schreiben.
 \begin{lstlisting}
  int a = (int) 5.3;
  System.out.println(a);
 \end{lstlisting}
 \pause
Java übernimmt das umwandeln. Manchmal muss man noch nichteinmal das machen.
 \begin{lstlisting}
  double a = 5;
  System.out.println(a);
 \end{lstlisting}
\end{frame}

\section{Datenkapselung}
\begin{frame}{Datenkapselung}
 Eines der grundlegenden Prinzipen der OOP.
 Daten sollen dahin, wo sie hingehören (und nur da!).
  Wenn möglich, geschieht dies nach dem \textbf{Geheimnisprinzip}. Jedes
  Objekt, jede Klasse ändert nur die Dinge, die ihr auch selbst gehören.
\end{frame}

\subsection{Sichtbarkeit}
\begin{frame}[fragile]{Sichtbarkeit}
 Um Programmierer dazu zu zwingen, die Datenkapselung einzuhalten,
 benutzen wir Sichtbarkeiten. Ist ein Attribut oder eine Methode als 
 \verb|private| gekennzeichent, so ist ein Zugriff nur innerhalb
 derselben Klasse möglich.
 Es gibt (in Java) drei Sichtbarkeiten:
 \begin{itemize}
  \item public
  \item protected
  \item private
 \end{itemize}
\end{frame}

\subsection{Packages}

\section{Mehr Stoff im Wiki!}
\subsection{Checkstyle}
\begin{frame}{Checkstyle}
 Insert interactive demonstration here
\end{frame}
\subsection{Eclipse}
\begin{frame}{Eclipse}
 Insert interactive demonstration here
\end{frame}
\section{Graphentheorie}
\begin{frame}{Was ist ein Graph?}
 Schwierige Frage.
 \begin{itemize}
  \item Eine Zeichnung mit Knoten und Kanten \pause
  \item Eine Operation auf einer Menge. \pause
  \item Eine Veranschaulichung einer Relation \pause
  \item Was ist eine Relation?
 \end{itemize}
\end{frame}

\subsection{Darstellung als Arrays}
\begin{frame}{Graphen als Arrays}
Kommt genauer im zweiten Semester. Hier dürft ihr das machen.
Ihr benötigt zwei Arrays. Eine speichert, wie viele Kanten von einer
Kante ausgehen. Die andere speichert die Ziele der Kanten.

\textbf{Aufgabe:} Entwerfe eine Klasse Graph, die einen Graphen speichern
kann. Die Knoten sollen hierbei von 0 durchnummeriert werden.
\end{frame}

\section{Matrizen}
\begin{frame}{Matrizen}
 Was ist eine Matrix? \pause
 
 Grob ist eine Matrix eine Tabelle. 
 Tabellen können als Arrays gespeichert werden. 
 
 \pause
 Verwende zweidimensionale Arrays!
\end{frame}

%%%% Ab hier muss nur noch das Bild gewechselt werden...
\section{Fragen zum Übungsblatt?}
\begin{frame}{Fragen?}
\end{frame}



%Lacher zum Schluss
\begin{frame}
 \includegraphics[scale=0.2]{../pics/11_waldo}
 
 \tiny{Quelle: 9gag.com}
\end{frame}


\end{document}